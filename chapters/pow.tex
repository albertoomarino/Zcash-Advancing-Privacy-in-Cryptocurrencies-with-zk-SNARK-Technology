\section{Proof-of-Work}

Proof-of-Work (PoW) refers to any algorithm that operates on the concept of requiring a client to perform some work, which is then verified by the network. Typically, the required work involves performing complex calculations that are later verified by the network. These algorithms aim to discourage Denial of Service attacks and other abuses of service, such as network spam.

\subsection{Equihash}

Equihash is a Proof-of-Work (\textbf{PoW}) algorithm used by Zcash (formerly known as Zerocash) for mining. The Equihash algorithm was chosen for its features that promote decentralization and mining security, making this resource-intensive and costly process more accessible to everyone, while simultaneously making the currency more secure against centralized attacks.

\subsection{Details of the Equihash Algorithm}

Equihash is a PoW algorithm based on the generalized birthday problem, requiring a significant amount of memory to solve. Specifically:
\begin{itemize}
    \item \textbf{Birthday Paradox}: Equihash is based on a variant of this problem. It requires finding two distinct sets of nonces that, when used as input to a hash function, produce a result containing a specific sequence of leading zeros in the hash (similar to Bitcoin);
    \item \textbf{Parameters Used}: Equihash requires certain parameters to function correctly.
    \begin{itemize}
        \item \textbf{Parameter n}: This parameter defines the hash output size. In Zcash, the \textbf{Blake2b} hash algorithm is used to produce the hash output;
        \item \textbf{Parameter k}: This parameter affects the memory requirement to solve the problem; higher values of k correspond to greater memory demands and thus greater computational power required.
    \end{itemize}
\end{itemize}

\subsection{Mining Process with Equihash}

The term \textbf{mining} \cite{mining} can be translated as "extracting," emphasizing how it is a digital process through which cryptocurrencies can be obtained. Mining is not the creation of money through digital coins but a complicated verification procedure generated by leveraging the computing power of a computer or specialized tools designed for this purpose. Anyone providing computing power for mining operations becomes a \textbf{miner}. The mining process can be broken down into successive steps:
\begin{itemize}
    \item \textbf{Preparation}: The miner generates a block header that includes various information, such as a nonce and the Merkle Root of the transactions in the block;
    \item \textbf{Nonce Generation}: The miner varies the nonce and calculates hashes using Blake2b;
    \item \textbf{Solution Search}: The miner looks for pairs of nonces that meet the Equihash condition. Specifically, it seeks two distinct sets of nonces that, when combined, produce a hash with a specific number of leading zeros;
    \item \textbf{Verification}: The found solution must be easily verifiable but hard to find. This balances the algorithm, making mining fair for all participants;
    \item \textbf{Confirmation and Addition to the Block}: Once a valid solution is found, the block is added to the blockchain, and the miner receives a reward for the block.
\end{itemize}

\subsection{Miner Rewards}

Zcash miners are compensated for their work in verifying transactions and protecting the network through a combination of block rewards and transaction fees.
\begin{itemize}
    \item \textbf{Block Rewards}: Each new block created on the Zcash blockchain comes with a block reward, initially set at 2.0 ZEC. This reward has been gradually reduced over time following a halving schedule, similar to Bitcoin, to 0.625 ZEC (June 2024).
    These block rewards incentivize miners to contribute their computing power to the network. The rewards are distributed among all miners who contributed their computing power to the creation of a new block;
    \item \textbf{Transaction Fees}: Miners can also earn transaction fees from users sending Zcash transactions. These fees are voluntary and set by the transaction sender. However, a minimum fee is enforced to prevent an overwhelming number of transactions from being sent to the Zcash network merely to add (useless) work to the network.
\end{itemize}

\subsection{Advantages of Equihash}

The use of Equihash, while merely an implementation choice, brings several advantages to Zcash:
\begin{itemize}
    \item \textbf{ASIC Resistance}: By requiring a significant amount of memory, Equihash makes the use of ASICs (Application-Specific Integrated Circuits designed for a single specific use, such as mining a specific cryptocurrency) less advantageous compared to GPUs or CPUs, promoting more decentralized Zcash mining;
    \item \textbf{Security}: The nature of the birthday problem ensures that finding a valid solution requires a significant amount of calculations, guaranteeing the security of the network;
    \item \textbf{Decentralization}: By favoring the use of common hardware, Equihash helps maintain a more decentralized network compared to other PoW algorithms dominated by ASICs (e.g., Bitcoin).
\end{itemize}